\clearpage

\setcounter{thefigs}{0}

\clearpage
\stepcounter{thefigs}
\begin{figure}
\figurenum{\fignum}
\plotone{spec_lrg.ps}
\caption{\label{spec_lrg} Best fit LRG spectral template
	(normalization is for a 1 $M_\odot$ galaxy located 10 pc away, or
	equivalently a $10^12$ $M_\odot$ galaxy located 10 Mpc away).
\end{figure}

\clearpage
\stepcounter{thefigs}
\begin{figure}
\figurenum{\fignum}
\plotone{sfh_lrg.ps}
\caption{\label{sfh_lrg} Star-formation history corresponding to LRG
	spectral template of Figure \ref{spec_lrg}. Top panel shows the
	number of stars formed per logarithmic time interval ($t$ is
	expressed in years, curve is normalized for a $10^{12}$ $M_\odot$
	galaxy). Almost all of the stars are formed in the first couple of
	billion years --- note that the recent ``spike'' is represents a
	tiny fraction ($\sim 10^{-8}$) of the total number of stars. Bottom
	panel shows the mean metallicity of the population as a function of
	time.  }
\end{figure}

\clearpage
\stepcounter{thefigs}
\begin{figure}
\figurenum{\fignum}
\plotone{lrg_colors.ps}
\caption{\label{lrg_colors} SDSS colors of LRGs as a function of
	redshift. The greyscale is the conditional distribution of color
	within each redshift bin. The thin lines are the 10\%, 25\%, 50\%,
	75\%, and 90\% quantiles of the distribution. The thick line
	is the prediction of the model. The $u$ band is not included in the
	fit, and the $u$ magnitudes of most LRGs are poorly known. The other
	colors fit the models reasonably well. This model, remember, is
	given incredible freedom, meaning that the above agreement is the
	best one can do with the stellar population synthesis code of
	\citet{bruzual03a}. }
\end{figure}

\clearpage
\stepcounter{thefigs}
\begin{figure}
\figurenum{\fignum}
\plotone{fullfits.ps}
\caption{\label{fullfits} Color residuals (defined explicitly in the
	text) of GALEX, SDSS, and 2MASS observations relative to our best
	fit 5-template model.  The greyscale is the conditional distribution
	of the color residual given the redshift.  The thin lines are the
	10\%, 25\%, 50\%, 75\%, and 90\% quantiles of the distribution. The
	thick dashed lines show the estimated 1$\sigma$ uncertainties in the
	colors from the photometric catalogs. Relative to the uncertainties,
	there are no significant biases or redshift trends in these fits. }
\end{figure}

\clearpage
\stepcounter{thefigs}
\begin{figure}
\figurenum{\fignum}
\plotone{goods.ps}
\caption{\label{goods} Color residuals in fit using the standard five
templates to GOODS data, compared to the typical uncertainties (thick
dashed lines). Note that the fits always do poorly on the $H$ band,
which we believe to be a catalog error. }
\end{figure}

\clearpage
\stepcounter{thefigs}
\begin{figure}
\figurenum{\fignum}
\plotone{goods_special.ps}
\caption{\label{goods_special} Same as Figure \ref{goods}, but fitting
using five templates specially designed for GOODS. These templates
have smaller residuals in many respects but still fail to fit the $H$
band data. }
\end{figure}

\clearpage
\stepcounter{thefigs}
\begin{figure}
\figurenum{\fignum}
\plotone{specfit.ps}
\caption{\label{specfit} Best fit model spectra based on the five
template fit to $g$, $r$ and $i$ fluxes, compared to the original
SDSS spectra from which we computed those fluxes.  The models and the
original spectra agree very well.}
\end{figure}

\clearpage
\stepcounter{thefigs}
\begin{figure}
\figurenum{\fignum}
\plotone{twomass_resid.ps}
\caption{\label{twomass_resid} Similar to Figure \ref{fullfits} but
for galaxies observed in both SDSS and 2MASS and only using SDSS and
2MASS bands. }
\end{figure}

\clearpage
\stepcounter{thefigs}
\begin{figure}
\figurenum{\fignum}
\plotone{twomass_resid.ps}
\caption{\label{twomass_resid} Similar to Figure \ref{fullfits} but
for galaxies observed in both SDSS and 2MASS and only using SDSS and
2MASS bands. }
\end{figure}
