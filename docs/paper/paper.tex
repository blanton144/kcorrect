\documentclass[10pt,preprint]{aastex}

\newcommand{\vv}[1]{{\bf #1}}
\newcommand{\df}{\delta}
\newcommand{\dfft}{{\tilde{\delta}}}
\newcommand{\betaft}{{\tilde{\beta}}}
\newcommand{\erf}{{\mathrm{erf}}}
\newcommand{\erfc}{{\mathrm{erfc}}}
\newcommand{\Step}{{\mathrm{Step}}}
\newcommand{\ee}[1]{\times 10^{#1}}
\newcommand{\avg}[1]{{\langle{#1}\rangle}}
\newcommand{\Avg}[1]{{\left\langle{#1}\right\rangle}}
\def\simless{\mathbin{\lower 3pt\hbox
	{$\,\rlap{\raise 5pt\hbox{$\char'074$}}\mathchar"7218\,$}}} % < or of order
\def\simgreat{\mathbin{\lower 3pt\hbox
	{$\,\rlap{\raise 5pt\hbox{$\char'076$}}\mathchar"7218\,$}}} % > or of order
\newcommand{\iras}{{\sl IRAS\/}}
\newcommand{\petroratio}{{{\mathcal{R}}_P}}
\newcommand{\petroradius}{{{r}_P}}
\newcommand{\petronumber}{{{N}_P}}
\newcommand{\petroratiolim}{{{\mathcal{R}}_{P,\mathrm{lim}}}}
\newcommand{\band}[2]{\ensuremath{^{#1}{#2}}}

\setlength{\footnotesep}{9.6pt}

\newcounter{thefigs}
\newcommand{\fignum}{\arabic{thefigs}}

\newcounter{thetabs}
\newcommand{\tabnum}{\arabic{thetabs}}

\newcounter{address}

%% You can insert a short comment on the title page using the command below.

\slugcomment{Submitted to \aj}

%% If you wish, you may supply running head information, although
%% this information may be modified by the editorial offices.
%% The left head contains a list of authors,
%% usually a maximum of three (otherwise use et al.).  The right
%% head is a modified title of up to roughly 44 characters.  Running heads
%% will not print in the manuscript style.

\shortauthors{Blanton {\it et al.} (2000)}
\shorttitle{Low Redshift Galaxy Evolution}

%% This paper uses runs 752/756, rerun 4
%% It uses redshifts from version 3 of spectro1d

%% This is the end of the preamble.  Indicate the beginning of the
%% paper itself with \begin{document}.

\begin{document}
 
%% LaTeX will automatically break titles if they run longer than
%% one line. However, you may use \\ to force a line break if
%% you desire.

\title{Modeling Galaxy Spectral Energy Distributions in the SDSS
Spectroscopic Survey}

%% Use \author, \affil, and the \and command to format
%% author and affiliation information.
%% Note that \email has replaced the old \authoremail command
%% from AASTeX v4.0. You can use \email to mark an email address
%% anywhere in the paper, not just in the front matter.
%% As in the title, you can use \\ to force line breaks.

%\author{Michael Blanton}
%\affil{NASA/Fermilab Astrophysics Center\\
%Fermi National Accelerator Laboratory, Batavia, IL 60510-0500}
%\author{\and lots and lots of other, probably more important people}
%\affil{Some Institution\\
%Somewhere, Some City, Some State or Province}
%\email{blanton@fnal.gov}

% Authorship determined by those I was directly involved with
% in performing this work as well as those responsible for the photo-z
% plates and those who have characterized the filter curves (since
% these might be publicly distributed). 
\author{
Michael R. Blanton\altaffilmark{\ref{NYU}},
%Tamas Budavari\altaffilmark{\ref{JHU}},
%Andrew J. Connolly\altaffilmark{\ref{Pitt}},
Istv\'an Csabai\altaffilmark{\ref{JHU}},
Mamoru Doi\altaffilmark{\ref{Tokyo}}, and
%Daniel Eisenstein\altaffilmark{\ref{Arizona}},
James E. Gunn\altaffilmark{\ref{Princeton}}
%David W. Hogg\altaffilmark{\ref{NYU}}, and
%David J. Schlegel\altaffilmark{\ref{Princeton}}
%Julianne Dalcanton\altaffilmark{\ref{UW}},
%Jon Loveday\altaffilmark{\ref{Sussex}},
%Michael A. Strauss\altaffilmark{\ref{Princeton}},
%Mark SubbaRao\altaffilmark{\ref{Chicago}},
%David H. Weinberg\altaffilmark{\ref{Ohio}},
%John E. Anderson, Jr.\altaffilmark{\ref{Fermilab}},
%James Annis\altaffilmark{\ref{Fermilab}},
%Neta A. Bahcall\altaffilmark{\ref{Princeton}},
%Mariangela Bernardi\altaffilmark{\ref{Chicago}},
%J. Brinkmann\altaffilmark{\ref{APO}},
%Robert J. Brunner\altaffilmark{\ref{Caltech}},
%Scott Burles\altaffilmark{\ref{Fermilab}},
%Larry Carey\altaffilmark{\ref{UW}},
%Francisco J. Castander\altaffilmark{\ref{Chicago}, \ref{Pyrenees}},
%Andrew J. Connolly\altaffilmark{\ref{Pitt}},
%Istv\'an Csabai\altaffilmark{\ref{JHU}},
%Douglas Finkbeiner\altaffilmark{\ref{Berkeley}},
%Scott Friedman\altaffilmark{\ref{JHU}},
%Joshua A. Frieman\altaffilmark{\ref{Fermilab}},
%Masataka Fukugita\altaffilmark{\ref{CosmicRay},\ref{IAS}},
%G. S. Hennessy\altaffilmark{\ref{USNO}},
%Robert B. Hindsley\altaffilmark{\ref{USNO}},
%Takashi Ichikawa\altaffilmark{\ref{Tokyo}},
%\v{Z}eljko Ivezi\'{c}\altaffilmark{\ref{Princeton}},
%Stephen Kent\altaffilmark{\ref{Fermilab}},
%G. R.~Knapp\altaffilmark{\ref{Princeton}},
%D. Q.~Lamb\altaffilmark{\ref{Chicago}},
%R. French Leger\altaffilmark{\ref{UW}},
%Daniel C. Long\altaffilmark{\ref{APO}},
%Robert H. Lupton\altaffilmark{\ref{Princeton}},
%Timothy A.~McKay\altaffilmark{\ref{Michigan}},
%Avery Meiksin\altaffilmark{\ref{Edinburgh}},
%Aronne Merelli\altaffilmark{\ref{Caltech}},
%Jeffrey A. Munn\altaffilmark{\ref{USNO}},
%Vijay Narayanan\altaffilmark{\ref{Princeton}},
%Matt Newcomb\altaffilmark{\ref{CarnegieMellon}},
%R. C. Nichol\altaffilmark{\ref{CarnegieMellon}},
%Sadanori Okamura\altaffilmark{\ref{Tokyo}},
%Russell Owen\altaffilmark{\ref{UW}},
%Jeffrey R.~Pier\altaffilmark{\ref{USNO}},
%Adrian Pope\altaffilmark{\ref{JHU}},
%Marc Postman\altaffilmark{\ref{STScI}},
%Thomas Quinn\altaffilmark{\ref{UW}},
%Constance M. Rockosi\altaffilmark{\ref{Chicago}},
%Donald P. Schneider\altaffilmark{\ref{PennState}}, 
%Kazuhiro Shimasaku\altaffilmark{\ref{Tokyo}},
%Walter A. Siegmund\altaffilmark{\ref{UW}},
%Stephen Smee\altaffilmark{\ref{Maryland}},
%Yehuda Snir\altaffilmark{\ref{CarnegieMellon}},
%Chris Stoughton\altaffilmark{\ref{Fermilab}},
%Christopher Stubbs\altaffilmark{\ref{UW}},
%Alexander S.~Szalay\altaffilmark{\ref{JHU}},
%Gyula P.~Szokoly\altaffilmark{\ref{Potsdam}},
%Aniruddha R.~Thakar\altaffilmark{\ref{JHU}},
%Christy Tremonti\altaffilmark{\ref{JHU}},
%Douglas L. Tucker\altaffilmark{\ref{Fermilab}},
%Alan Uomoto\altaffilmark{\ref{JHU}},
%Dan vanden Berk\altaffilmark{\ref{Fermilab}},
%Michael S. Vogeley\altaffilmark{\ref{Drexel}},
%Patrick Waddell\altaffilmark{\ref{UW}},
%Brian Yanny\altaffilmark{\ref{Fermilab}},
%Naoki Yasuda\altaffilmark{\ref{NAOJ}},
%and Donald G.~York\altaffilmark{\ref{Chicago}}
}

\altaffiltext{1}{Based on observations obtained with the
Sloan Digital Sky Survey} 
\setcounter{address}{2}
\altaffiltext{\theaddress}{
\stepcounter{address}
New York University, Department of Physics, 4 Washington Place, New
York, NY 10003
\label{NYU}}
\altaffiltext{\theaddress}{
\stepcounter{address}
Department of Physics and Astronomy, The Johns Hopkins University,
Baltimore, MD 21218
\label{JHU}}
%\altaffiltext{\theaddress}{
%\stepcounter{address}
%University of Pittsburgh,
%Department of Physics and Astronomy,
%3941 O'Hara Street,
%Pittsburgh, PA 15260
%\label{Pitt}}
%\altaffiltext{\theaddress}{
%\stepcounter{address}
%rinceton University Observatory, Princeton,
%NJ 08544
%\label{Princeton}}
%\addtocounter{address}{1}
%\altaffiltext{\theaddress}{
%\stepcounter{address}
%Fermi National Accelerator Laboratory, P.O. Box 500,
%Batavia, IL 60510
%\label{Fermilab}}
%\altaffiltext{\theaddress}{
%\stepcounter{address}
%Department of Astronomy, University of Washington,
%Box 351580,
%Seattle, WA 98195 
%\label{UW}}
%\altaffiltext{\theaddress}{
%\stepcounter{address}
%University of Chicago, Astronomy \&
%Astrophysics Center, 5640 S. Ellis Ave., Chicago, IL 60637
%\label{Chicago}}
%\altaffiltext{\theaddress}{
%\stepcounter{address}
%Hubble Fellow 
%\label{Hubble}}
%\altaffiltext{\theaddress}{
%\stepcounter{address}
%Sussex Astronomy Centre,
%University of Sussex,
%Falmer, Brighton BN1 9QJ, UK
%\label{Sussex}}
%\altaffiltext{\theaddress}{
%\stepcounter{address}
%Ohio State University,
%Department of Astronomy,
%Columbus, OH 43210
%\label{Ohio}}
%\altaffiltext{\theaddress}{
%\stepcounter{address}
%Apache Point Observatory,
%2001 Apache Point Road,
%P.O. Box 59, Sunspot, NM 88349-0059
%\label{APO}}
%\altaffiltext{\theaddress}{
%\stepcounter{address}
%Department of Astronomy, California Institute of Technology,
%Pasadena, CA 91125
%\label{Caltech}}
%\altaffiltext{\theaddress}{
%\stepcounter{address}
%Observatoire Midi-Pyr\'en\'ees, 
%14 ave Edouard Belin, Toulouse, F-31400, France
%\label{Pyrenees}}
%\altaffiltext{\theaddress}{
%\stepcounter{address}
%Department of Astronomy and Research Center for 
%the Early Universe,
%School of Science, University of Tokyo,
%Tokyo 113-0033, Japan
%\label{Tokyo}}
%\altaffiltext{\theaddress}{
%\stepcounter{address}
%UC Berkeley, Dept. of Astronomy, 601 Campbell Hall, Berkeley, CA  94720-3411
%\label{Berkeley}}
%\altaffiltext{\theaddress}{
%\stepcounter{address}
%Institute for Cosmic Ray Research, University of
%Tokyo, Midori, Tanashi, Tokyo 188-8502, Japan
%\label{CosmicRay}}
%\altaffiltext{\theaddress}{
%\stepcounter{address}
%Institute for Advanced Study, Olden Lane,
%Princeton, NJ 08540
%\label{IAS}}
%\altaffiltext{\theaddress}{
%\stepcounter{address}
%U.S. Naval Observatory,
%3450 Massachusetts Ave., NW,
%Washington, DC  20392-5420
%\label{USNO}}
%\altaffiltext{\theaddress}{
%\stepcounter{address}
%University of Michigan, Department of Physics,
%500 East University, Ann Arbor, MI 48109
%\label{Michigan}}
%\altaffiltext{\theaddress}{
%\stepcounter{address}
%Department of Physics \& Astronomy,
%The University of Edinburgh,
%James Clerk Maxwell Building,
%The King's Buildings,
%Mayfield Road,
%Edinburgh EH9 3JZ, UK
%\label{Edinburgh}}
%\altaffiltext{\theaddress}{
%\stepcounter{address}
%Department of Physics, Carnegie Mellon University, 
%5000 Forbes Avenue, Pittsburgh, PA 15213-3890 
%\label{CarnegieMellon}}
%\altaffiltext{\theaddress}{
%\stepcounter{address}
%Space Telescope Science Institute, Baltimore, MD 21218
%\label{STScI}}
%\altaffiltext{\theaddress}{
%\stepcounter{address}
%Department of Astronomy and Astrophysics,
%The Pennsylvania State University,
%University Park, PA 16802
%\label{PennState}}
%\altaffiltext{\theaddress}{
%\stepcounter{address}
%Department of Astronomy,
%University of Maryland,
%College Park, MD 20742-2421 
%\label{Maryland}}
%\altaffiltext{\theaddress}{
%\stepcounter{address}
%Astrophysikalisches Institut Potsdam,
%An der Sternwarte 16, D-14482 Potsdam, Germany
%\label{Potsdam}}
%\altaffiltext{\theaddress}{
%\stepcounter{address}
%Department of Physics, Drexel University, Philadelphia, PA 19104
%\label{Drexel}}
%\altaffiltext{\theaddress}{
%\stepcounter{address}
%National Astronomical Observatory, Mitaka, Tokyo 181-8588, Japan
%\label{NAOJ}}
%\addtocounter{address}{1}
%\altaffiltext{\theaddress}{Physics Dept., University of California, Davis, CA 95616
%\label{UCDavis}}
%\addtocounter{address}{1}
%\altaffiltext{\theaddress}{IGPP/Lawrence Livermore National Laboratory
%\label{IGPP}}
%\addtocounter{address}{1}
%\altaffiltext{\theaddress}{Department of Astronomy, University of California, Berkeley, C
%A 94720-3411
%\label{Berkeley}}
%\stepcounter{address}
%\altaffiltext{\theaddress}{Remote Sensing Division, Code 7215, Naval
%Research Laboratory, Washington, DC 20375
%\label{NRL}}
%\addtocounter{address}{1}
%\altaffiltext{\theaddress}{U.S. Naval Observatory, Flagstaff Station,
%P.O. Box 1149,
%Flagstaff, AZ  86002-1149
%\label{Flagstaff}}

\clearpage

%% Mark off your abstract in the ``abstract'' environment. In the manuscript
%% style, abstract will output a Received/Accepted line after the
%% title and affiliation information. No date will appear since the author
%% does not have this information. The dates will be filled in by the
%% editorial office after submission.
\begin{abstract}
We present a method of inferring galaxy spectral energy distributions
(SEDs) from photometric observations of galaxies in broad band
filters. Using the Sloan Digital Sky Survey (SDSS) observations in the
optical wavelength regime, we demonstrate that the method yields
robust results. We show how the method can be used to recover galaxy
magnitudes in fixed frame bandpasses over a range of redshifts. We
compare our method to others, including galaxy spectrophotometry. For
the SDSS, fixed frame magnitudes around the SDSS $r$, $i$, and $z$
bands are highly reliable, at the level of a few percent; $g$-band
fixed frame magnitudes are somewhat more uncertain, and $u$-band fixed
frame magnitudes are highly uncertain, at the level of 10--20\%. All
code and templates from this paper are public.
\end{abstract}

\keywords{galaxies: fundamental parameters --- galaxies: photometry
--- galaxies: statistics}

%
% Introduction and motivation
%

\section{Motivation}
\label{motivation}

Photometric surveys of the sky observe galaxies in bandpasses which
are fixed in the rest-frame of the observer. However, spectra of
distant galaxies are redshifted due to the expansion of the
universe. Thus, the more distant the galaxy, the further the observed
bandpass is blueshifted relative to the rest-frame spectral energy
distribution (SED) of the object observed. To compare observations of
galaxies at different redshifts, it is important to account for the
fact that one is observing different parts of the galaxy SED.

In the past, this accounting has in general come in the form of
``$K$-corrections'' applied to the observed magnitudes. That is, a
function $K(z)$ is added to the standard cosmological bolometric
distance modulus $\mathrm{DM}(z)$. Sometimes a single function has
been applied regardless of the galaxy type, though more recently it
has become standard to use a discrete set of $K(z)$ functions
depending on galaxy type (based either on morphology or spectral
features). As described in \cite{oke68a} and later papers, the
function is based on the projection of an assumed galaxy SED
redshifted to $z$ onto the filter response.

However, galaxies do not all have the same SED, nor are they selected
from some discrete set of SEDs. For this reason, schemes for applying
the $K$-corrections can fail to be self-consistent: the SED assumed
for the $K$-corrections can be significantly inconsistent with the
observed galaxy colors! As we demand more precision from our
astronomical data analysis in new, large multi-band surveys, and in
particular as we try to quantify the evolution of galaxies, we must
take a more sophisticated approach to approximating fixed-frame
observations of galaxies.

Our approach here is to develop a method for inferring the underlying
SEDs of a set of galaxies at a range of redshifts by requiring that
their SEDs all be drawn from a similar population. For each galaxy, we
will recover a model SED, which can be used to synthesize the galaxy's
magnitude in any bandpass. In addition, this method provides a natural
way to infer an estimate of the flux within the whole optical range.
The approach is equivalent to the photometric metric redshift
estimation methods of \citet{csabai99a} and \citet{budavari00a}, 
except we use slightly different coordinate systems.

Section \ref{sedfit} describes our method of fitting galaxy SEDs to
broad-band photometry. An implementation of the method in C and IDL is
provided at a URL on the World-Wide Web, through a public CVS
repository, or by request.  Section \ref{data} applies the method to
galaxies in the SDSS, showing that the fits are robust. Section
\ref{kcorrection} shows how one derives $K$-corrections from the
results, and discusses how best to use the results of such
fits. Section \ref{conclusion} concludes and discusses future
development of the method described here.

\section{Fitting SEDs to Galaxy Broad-band Colors}
\label{sedfit}

Our task is to recover a model for the galaxy SED from broad-band
photometric measurements. Since a set of broad band galaxy fluxes does
not correspond uniquely to a particular SED, and because galaxy SEDs
are known to have significant structure over wavelength ranges small
compared to our bandpass, this task is an ill-posed, inverse
problem. However, we are not completely ignorant about the forms which
galaxy SEDs take, so we can attempt to use what we know about galaxy
SEDs to simplify our task.

The method described here for doing so follows closely the methods of
estimating photometric redshifts used by \citet{csabai00a} and
\citet{budavari00a}. It is designed to take advantage of what we
already understand about galaxy spectral energy distributions.
%In addition, as implemented here, the method is
%equivalent to a fit to the star-formation history of the galaxies
%using a spectrophotometric model; we will investigate the implications
%of the model fits in a separate paper and concentrate here on the
%empirical aspects of the results.
Beginning with a space defined by $N_b$ template galaxy SEDs
$\vv{v}_i(\lambda)$ (say from \citealt{bruzual93a}), one can find an
orthonormal set of spectra spanning that space. To be more precise, we
define the dot-product in the space of galaxy spectra to be:
\begin{equation}
\vv{b}_i \cdot \vv{b}_j =
\int_{\lambda_{\mathrm{min}}}^{\lambda_{\mathrm{max}}} d\lambda
\lambda b_i(\lambda) b_j(\lambda),
\end{equation}
where $\lambda_{\mathrm{min}}$ and $\lambda_{\mathrm{max}}$ define the
wavelength range used to define orthogonality. These limits are
defined not over the full range covered by the $\vv{v}_i(\lambda)$
spectra, but instead over a smaller range corresponding to the range
which is within the observed wavelength range over most of the
redshift range of the sample. Otherwise, orthogonality is defined in
terms of regions of the spectra which are poorly constrained, which
make interpretation of the fits inconvenient.  This choice does not
affect the results for the reconstructed SED fit quantitatively
whatsoever.\footnote{If the template spectra were identical in the
subrange chosen and only differed outside of it, this choice would
cause computational problems, but in fact the spectra are all
independent in the restricted, as well as the full, wavelength range.}

We then further define the basis of the SED space such that the $N_b$
template SEDs, $\vv{v}_i$, are linear combinations of $N_b$ basis
spectra $\vv{b}_i$ and that $\vv{b}_i \cdot \vv{b}_j =
\delta_{ij}$. These conditions do not fully specify the $\vv{b}_i$,
which naturally may be rotated arbitrarily within our
$N_b$-dimensional subspace of spectral space.  In our implementation,
we use the subspace defined by linear combinations of ten
Bruzual-Charlot instantaneous burst models (our $\vv{v}_i(\lambda)$)
with ages ranging from $3 \times 10^7$ to $2\times 10^{10}$ years,
five with metallicity $Z=0.02$ and six with $Z=0.004$, all assuming a
Salpeter Initial Mass Function and smoothed using a Gaussian with a
standard deviation of $\sigma = 300$\AA. The smoothing is necessary
to prevent the method from overfitting; not all the small-scale
information about the templates can be recovered using this method,
and we find that this level of smoothing gives sensible results.

We cannot determine an individual galaxy's position in this
ten-dimensional space from its broad-band colors alone, because there
are only five observed bands. However, as outlined by
\citet{csabai00a}, one can find a lower dimensional space defined by
$N_t$ vectors $\vv{e}_k$ which best fits the {\it set} of
galaxies. The SED of galaxy $i$ may be approximated as a linear
combination of the basis spectra in this lower dimensional space:
\begin{equation}
\label{lincomb}
f_{\mathrm{model},i}(\lambda) d\lambda = 
\sum_k a_{i,k} \sum_j e_{k,j} b_j(\lambda) d\lambda,
\end{equation}
where $a_k$ are the components of the galaxy projected in the
low-dimensional space. From this SED it is easy to determine the flux
$F_{\mathrm{model},il}$ measured in each survey bandpass by projecting
the model SED onto the bandpass $l$. We define
\begin{equation}
\chi^2 = \sum_i \sum_l
\frac{(F_{\mathrm{obs},il}-F_{\mathrm{model},il})^2}{\sigma_{il}^2},
\end{equation}
where $\sigma_{il}$ is the estimated error in the observed flux
$F_{\mathrm{obs},il}$ in bandpass $l$ for galaxy $i$. Taking
derivatives of $\chi^2$ with respect to $a_{i,k}$ and $e_{k,j}$
results in a set of equations bilinear in $a_{i,k}$ and $e_{k,j}$.
\citet{csabai00a} describe how to iteratively solve this bilinear
equation by first fixing $e_{k,j}$ and fitting for $a_{i,k}$, then
fixing $a_{i,k}$ and fitting for $e_{k,j}$, and iterating that
procedure.

%We put one additional constraint on our SEDs. To regularize the edges
%of the SEDs, we define two additional bands blueward and redward of
%$u$ and $z$, respectively, which we will call $a_1$ and $a_2$. The
%bandpasses are simply Gaussians centered at 2700\AA\ and 10200\AA\
%with standard deviations of 250\AA\ and 350\AA, respectively. From
%studying synthetic galaxy SEDs redshifted between $z=0$ and $z=0.2$,
%it is clear that the $u-a_1$ color varies not much more than 1.5
%magnitudes around the value 1.5, and that the $z-a_2$ color usually
%varies only by about 0.2 magnitudes around 0.1. Thus, given the $u$
%and $z$ magnitude, we can put a very weak constraint on the $a_1$ and
%$a_2$ magnitudes.  Similarly, when one of the SDSS bands is
%unobserved, we replace it based on an adjacent band and the mean color
%for galaxies in the sample, with a large error bar. These weak
%constraints prevent the inferred galaxy SEDs from becoming too
%unreasonable in the regions of the spectrum without data.

We define the ``flux'' $f_k$ of each template as the flux in the range
$\lambda_{\mathrm{min}}<\lambda<\lambda_{\mathrm{max}}$; then an
estimate of the flux in this range is $F_{{v}}=\Sigma_k a_k f_k$. The
locus of points of a fixed flux in the component space $a_k$ lie on a
plane. Therefore, a natural choice for the orientation of the axes
$\vv{e}_k$ is to let $\vv{e}_0$ be perpendicular to the planes of
constant flux; in this manner, the coefficient $a_0$ in Equation
(\ref{lincomb}) is directly proportional to $F$ and is thus purely a
measure of the object's flux (in the wavelength range considered),
while the parameters $a_i/a_0$ (where $i>0$) primarily measure the
shape of the galaxy SED in the wavelength range considered. $F$ can be
trivially related to $L$, the luminosity in this range, by the inverse
square of the luminosity distance.  As it turns out, the parameters
$a_i/a_0$ tend to be distributed very approximately in an ellipsoid,
with not much curvature within the space defined by the template
SEDs. Thus, we rotate the axes $\vv{b}_i$ (for $i>0$) such that they
are aligned with the principal axes of the ellipsoid. This choice of
coordinate system is arbitrary, however, and does not affect our
quantitative results.

In this way, we can reconstruct SEDs based on the broad-band
magnitudes of galaxies. From these reconstructed SEDs one can estimate
$K$-corrections, develop a measure of galaxy spectral type, or to
synthesize a very broad-band measure of the galaxy flux.  The
templates determined during the procedure can, of course, be used to
estimate photometric redshifts, as \citet{csabai00a} and
\citet{budavari00a} describe.

The version of the $K$-correction code (v1\_0) implementing the
calculations described here, along with eigentemplates and filter
curves is publicly available on the World Wide Web at {\tt
http://physics.nyu.edu/\~mb144/kcorrect}. In addition, a public CVS
repository can be accessed at {\tt xxx} (instructions for exporting
the appropriate version of the code from this repository are provided
on the web page noted above. The whole of the code can be used through
the Research Systems, Incorporated, IDL language; everything except
for the template-fitting also exists in a stand-alone C version (which
calls the same routines, guaranteeing consistency).

\section{Application to SDSS Data}
\label{data}

In this section we describe the SDSS data set, we describe how we fit
the templates and the coefficients in detail for this data set, and we
describe our distribution of code to do this.

\subsection{SDSS Spectroscopic Data}

The SDSS (\citealt{york00a}) is producing imaging and spectroscopic
surveys over $\pi$ steradians in the Northern Galactic Cap. A
dedicated 2.5m telescope (Siegmund {\it et al.}, in preparation) at
Apache Point Observatory, Sunspot, New Mexico, images the sky in five
bands between 3000 and 10000 \AA\ ($u$, $g$, $r$, $i$, $z$;
\citealt{fukugita96a}) using a drift-scanning, mosaic CCD camera
(\citealt{gunn98a}), detecting objects to a flux limit of $r'\sim
22.5$. The ultimate goal is to spectroscopically observe 900,000
galaxies, (down to $r_{\mathrm{lim}}'\approx 17.65$), 100,000 Bright
Red Galaxies (BRGs; Eisenstein {\it et al.}, in preparation), and
100,000 QSOs (\citealt{fan99a}; Newberg {\it et al.}, in preparation)
will be targeted for spectroscopic follow up using two digital
spectrographs on the same telescope. Many of the details of the survey
are described in a paper accompanying the Early Data Release
(\citealt{stoughton01a}). The survey has begun in earnest, and has
obtained about 20\% of its intended data.

As of October 2001, the SDSS had imaged and targeted around 1,739
square degrees of sky and taken spectra of approximately 170,000
objects over 1,169 square degrees of that area. As described below,
only a subset of these galaxies are used here to develop the
templates, though we calculate the SED fits for all of them. The
results of the photometric pipeline for all of these objects were
extracted from the SDSS Operational Database {\bf ref munn?}. The
photometry used here for the bulk of these objects was the same as
that used when the objects were targeted.  However, for those objects
which were in the EDR photometry, we used the better calibrations and
photometry from the EDR.

The magnitudes are calibrated $AB$, meaning that the system is
designed such that:
\begin{eqnarray}
m_{AB} &=& ZZZZ - 2.5 \log_{10}\left[
\frac{\int_{0}^{\infty} d\lambda \lambda f(\lambda) R(\lambda)}
{\int_{0}^{\infty} d\lambda \lambda^{-1} R(\lambda)}\right]
\cr
&=& ZZZZ - 2.5 \log_{10}\left[
\frac{\int_{0}^{\infty} d\nu \nu^{-1} f(\nu) R(\nu)}
{\int_{0}^{\infty} d\nu \nu^{-1} R(\nu)}\right],
\cr
\end{eqnarray}
where $R(\lambda)$ is the fraction of photons which are detected as a
function of wavelength (a unitless quantity). 

The response functions $R(\lambda)$ have been measured using a
monochrometer by {\bf how ref Doi}.  Using a model for the atmospheric
transmission and the reflectivity of the primary and secondary
mirrors, one can then model the response of the entire system. For
each filter in the SDSS, there are six different CCDs; it has been
shown that the differences between these six camera columns are
negligible. The resulting set of filter curves is shown in Figure
\ref{response_sdss}, in comparison to a model of a galaxy spectral
energy distribution observed at $z=0$ (a 4 Gyr old instantaneous burst
from the models of \citealt{bruzual93a}).

We extract one-dimensional spectra from the two-dimension images using
a pipeline created specifically for the SDSS instrumentation
(\citealt{schlegel02a}). We also use redshifts determined by the same
pipeline; the official SDSS redshifts are obtained from a different
pipeline (\citealt{subbarao02a}). The two independent versions provide
a consistency check on the redshift determination. They are consistent
(for galaxies) at over the 99\% level.\footnote{The official pipeline
tends to work better for objects with unusual spectra, such as certain
types of stars and QSOs.}.

We use throughout the magnitudes determined by the SDSS using the
modified form of the magnitude described by \citet{petrosian76a}, as
described in \citet{strauss02a}, reddening-corrected using the dust
maps of \citet{schlegel98a}. These magnitudes have a nearly constant
metric aperture as a function redshift; in addition, all the bands use
the same aperture, so the measured SED corresponds (to within the
effects of seeing) to the SED of an identifiable region of the galaxy.

\subsection{Fitting SEDs to SDSS Data}

For the purposes of applying the method described in Section
\ref{sedfit} to SDSS data, we take a subsample of the data consisting
of around 25,000 objects in the range $0.0<z<0.5$. The sample includes
both the main sample and the LRGs; we sample these spectra such that
we get an approximately even distribution within our range of
redshifts. In addition, we add results from galaxies in several plates
(totalling about 1,000 objects) which were selected by a photometric
redshift algorithm ({\bf ref who?}) to be at around $z\sim 0.3$. These
objects are invaluable for tying down the blue end of the templates.

Some of the objects have missing or poorly constrained data. For
example, the $u$- or $z$-band fluxes for some objects are swamped by
the photon noise of the sky. We identify such cases as magnitude
errors greater than 0.8 or magnitudes fainter than 22.5 in any
band. We ignore these objects entirely when fitting for the
templates. However, we still want to determine a best-fit SED for each
object. For this purpose, we replace a ``bad'' magnitude based on an
adjacent (usually the redder) band and the average color in those
bands of objects at that redshift. We give these measurements a large
error bar, of 0.8 magnitudes. Thus, we account for the missing
information by simply requiring that the object SED have
``reasonable'' properties.

We choose to fit for $N_t = 4$ eigentemplates, the maximum one can use
and still fit allow freedom to fit for the templates themselves. Since
galaxies live in a low-dimensional space, we find it is worth being
able to fit for the templates. Simply using five templates (which
obviously reproduces all the magnitudes exactly) tends to yield
unphysical trends of galaxy SED versus redshift ({\it cf.}  Figure
\ref{constantz} below). Using three templates does nearly as well as
four templates in the sense that the resulting templates reproduce the
magnitudes nearly as well. However, since one of the applications of
these SED determinations is the distribution of galaxy colors in
fixed-frame magnitudes, we don't want to artificially reduce the
dimensionality of the color space to only two.

One more choice needs to be made, the wavelength regime over which to
orthogonalize the templates and over which to calculate the flux. We
choose the range defined by $\lambda_{\mathrm{min}}=3500$\AA and
$\lambda_{\mathrm{max}}=7500$\AA, since this range is included for
almost all galaxies in the sample. We refer here and in other papers
to the flux and luminosity in this range as the ``visual flux'' $f_v$
and the ``visual luminosity'' $L_v$.

\subsection{Results}

The resulting four templates are shown in Figure \ref{templates} in
the range $2000<\lambda<12000$\AA, the region which is constrained by
our observations. Figure \ref{coeffs} is a more illuminating
figure. It shows the pairwise joint distributions of $a_1/a_0$,
$a_2/a_0$, and $a_3/a_0$ in the top three panels. In the bottom panel,
three spectra taken from a one-dimensional sequence along the galay
locus are shown, showing that the spectra become progressively bluer
along that sequence.

These reconstructed spectra do an excellent job of reproducing the
observed galaxy fluxes. Figure \ref{model} shows the differences
between the observed and reconstructed fluxes as a function of
redshift and as a function of absolute magnitude (in $\band{0.1}{r}$).
The solid lines show the median differences. The estimates of the
fluxes in $g$, $r$, $i$, and $z$ do not correlate strongly with any of
these parameters, while for $u$ there is a 5--10\% discrepancy for the
most distant galaxies, in the sense that the model $u$-band flux is
too bright. The dotted lines show the boundaries containing 80\% of
the objects surrounding the median (roughly the $1.3\sigma$ limits if
the distribution were Gaussian). For $g$, $r$, and $i$, the scatter is
very small --- certainly comparable to the errors in calibration ---
and roughly independent of redshift and absolute magnitude. For $z$,
the scatter is somewhat larger, perhaps reflecting the poor
signal-to-noise in $z$, but remains constant with redshift. For $u$,
the scatter is larger, perhaps reflecting the fact that four templates
simply is not enough to characterize galaxy SEDs simultaneously in the
optical and near ultra-violet. After all, galaxies are known to have
large variations in their near-ultraviolet properties (because it
depends strongly on dust content and recent star-formation). This
effect may become more important with redshift as $u$ probes bluer and
bluer regions of the SED. All told, these spectra seem to be fair
estimates of the true underlying spectra.

An important test of the reasonableness of the fits is to check that
for a fixed absolute magnitude, the distribution of the SED fits
depends little on redshift. Figures \ref{coeffs1}--\ref{coeffs4}
demonstrates this to be the case for galaxies in several luminosity
ranges. The left column of Figure \ref{coeffs1} shows the three
coefficient ratios $a_i/a_0$ for each galaxy as a function of
redshift. The right column shows the distribution of these coefficient
ratios for two redshift ranges, showing that they are nearly
consistent. The resulting change in the color distribution inferred
from these coefficients is similarly small, at the level of what one
would expect intrinsic evolution to cause.

In short, these fits to galaxy SEDs provide estimates which reproduce
the galaxy photometry nearly to the level of the errors in the
photometry itself, seem physically reasonable, and are consistent over
the range of redshifts we consider ($0.0. < z < 0.5$).

\section{$K$-corrections}
\label{kcorrection}

\subsection{Calculating $K$-corrections}

With a model of the rest-frame galaxy SED in hand, it is 
straightforward to project that SED onto a filter corresponding to 
an observation at any redshift, and predict the so-called
$K$-correction to the galaxy magnitude in order to compare galaxies 
at different redshifts. One simply calculates the AB magnitude at 
any redshift as:
\begin{equation}
m_{AB} = ZZZZ - 2.5 \log_{10}\left[
\frac{\int_{0}^{\infty} d\lambda \frac{\lambda}{1+z}
f(\frac{\lambda}{1+z}) R(\lambda)}
{\int_{0}^{\infty} d\lambda \lambda^{-1} R(\lambda)}\right].
\end{equation}
The simple way to think about this equation is that simply moving away
from the Earth doing does not change the bolometric flux of the
galaxy. So when the spectrum is stretched by the $(1+z)$ redshift
factor, as in $f(\lambda/(1+z))$, the bolometric flux should be
constant; that is, you must apply the extra multiplicative factor
$1/(1+z)$ to counteract the stretching of the spectrum. Once you have
have accounted for the fact that you are observing a particular
bandpass given the redshift of the object, you can then go ahead and
apply to the calculated magnitude the cosmological distance modulus
(that is, the luminosity distance-squared law, as tabulated by, {\it
e.g.} \citealt{hogg97a}).

We want to emphasize here that, while $K$-correcting to a fixed frame
bandpass is sometimes necessary in order to achieve a scientific
objective, it is not always necessary or appropriate. Because
$K$-corrections are an inherently uncertain business (the broad-band
magnitudes just do not uniquely determine the SED) they should be
avoid or minimized when possible. For example, if one was calculating
the evolution of clustering of red and blue galaxies separately, it
would perhaps be wise to perform the separation not on the
$K$-corrected colors of the galaxies but on the median color of, say,
$M_\ast$ galaxies as a function of redshift. Similarly, in situation
where $K$-corrections cannot be ignored, such as the calculation of
the evolution of the luminosity function, their effect should be
minimized by, for example, correcting to the median redshift of
galaxies in the sample.

For this reason, we here define a notation which we will use in
subsequent papers for the effective rest-frame bandpass at any
redshift in band $b$, namely:
\begin{equation}
\band{z}{b}
\end{equation}
read, ``$b$-band at redshift $z$.'' Now, typically one's measurements
in this band will be difficult to connect to, say, solar luminosities
in $\band{z}{b}$, simply because people have not projected the appropriate
stellar spectrophotometry onto these bandpasses. However, it is our
position that stars are {\it better} understood than galaxies, and
that it therefore is in the end simpler to stay as close as possible
to the system in which the galaxies are observed. In any case,
astronomy is quickly reaching a level of precision for which the exact
nature of the bandpasses used has to be known and considered in most
analyses of observational data.

The reader may ask why, if we are basing our $K$-corrections on a full
model of the galaxy SED, we do not correct to bandpasses with simpler
shapes (say, Gaussians). The reason is that we want the effective
bandpasses to actually have been observed for some galaxies in the
sample, so that the $K$-correction for those galaxies is formally and
actually zero. This property is desirable because, as noted above,
there are uncertainties in the $K$-corrections.

\subsection{Testing the $K$-corrections}

We show, in Figure \ref{kcorrect}, the $K$-corrections to $z=0.1$
inferred from this method, for all five SDSS bands. Note that the
$K$-corrections are largest (and thus most uncertain) in
$\band{0.1}{u}$ and $\band{0.1}{g}$. 

To test how robust these $K$-correction results are to our model
assumptions, we compare them to other methods of calculating
fixed-frame galaxy magnitudes.  An extremely simple method is to
calculate the flux in any desired bandpass by fitting a power-law
slope and amplitude to the fluxes in the two adjacent bandpasses
(extrapolating when necessary). Figure \ref{ciCompare.sample8b15}
shows the differences in the $K$-corrections inferred from this method
and those inferred from the method of Figure
\ref{kcorrect.sample8b15}, as a function of redshift. $r$, $i$, and
$z$ are all reasonably similar in either method; $u$ and $g$, however,
have distinct trends with redshift, mostly due to the sharper
gradients in the spectra in this regime. To show this fact, we perform
a similar power-law fit, only this time including a break in the
spectrum at 4000\AA. We use the $u$-$g$ color to fit the break,
assuming that the slope blueward of 4000\AA\ is $f(\lambda)\propto
\lambda^{2}$.  Figure \ref{cibreakCompare.sample8b15} shows the
results of this fit; the redshift trend in $g$ is greatly reduced, as
is the trend in $u$, but a large amount of scatter remains in the $u$
band. This results from the fact that you can {\it either} fit the
slope of the SED below the 4000 \AA\ break {\it or} the size of the
4000 \AA\ break itself; it is not possible with this data to constrain
both in an individual spectrum, which is a fundamental limitation of
our analysis of a fixed frame $u$-band luminosity function.

Finally, it is possible to use the galaxy spectra obtained with the
spectrograph to estimate the $K$-correction for each object (in
\band{0.1}{g}, \band{0.1}{r}, and \band{0.1}{i}). However, this
procedure requires trusting the spectrophotometry over a wide
wavelength range. In addition, the spectra only probe the inner parts
of galaxies, and because the fiber aperture is fixed in angular space,
a different region of the galaxy is probed as a function of
redshift. Nevertheless, in Figure \ref{specK}, we compare the
$K$-corrections calculated based on the spectra to those of Figure
\ref{kcorrect}, finding that they are extremely similar. The fact that
they are so similar might prove useful, in the sense that one could
combine the photometric and spectroscopic results, using the spectra
to constrain small-scale features such as the 4000 \AA\ break and the
photometry to cover a large wavelength range and to probe the full
extent of the galaxy. Such a procedure might resolve the difficulties
in handling the $u$-band discussed in the previous paragraph. We will
not pursue this issue further in this paper and leave it for future
investigation.

\section{Conclusions and Future Work}
\label{conclusions}

We have presented a method and an implementation for estimating galaxy
SEDs for the purpose of calculating fixed-frame galaxy magnitudes over
a range of redshifts. We have demonstrated that it gives sensible and
consistent results. We will be using this method in future papers
which will describe the joint distribution of luminosities and colors
of galaxies, as well as the evolution of the luminosity function of
galaxies. Furthermore, we plan to incorporate observations of objects
in bands other than the SDSS bands to further describe the nature of
galaxy SEDs.

Some of the scatter in the three-dimension space describing the shape
of the galaxy SED is probably due to dust. It may be possible to
evaluate the effects of reddening in this space and, by assuming that
galaxies corrected for internal reddening live in an even lower
dimensional space, perform a reddening correction. We will be
investigating this question in the near future.

Furthermore, one may want to include the effects of evolution in the
$K$-corrections. One approach to this calculation is to define a
one-dimensional locus, and record which position on this locus is
closest to each galaxy. Then, the one-dimensional locus provides a set
of spectra, for which estimates of evolution can be calculated. This
procedure avoids fitting star-formation histories to a huge number of
different positions in galaxy SED space. Since their will be
degeneracies between age and metallicity in these fits in any case, it
may be best to simplify the problem to one dimension.

The distribution of galaxies in the three-dimensional space of galaxy
SEDs described here may be useful for other purposes, as well. As one
example, which has already been explored in a separate project, one
can calculate photometric redshifts (if you are willing to restrict
your galaxy models to a one-dimensional, though possibly curved, locus
within the three-dimensional space). However, there is a fairly severe
redshift-type degeneracy, in the sense that blue galaxies at high
redshift can be confused with red galaxies at lower redshift. These
problems can be hazardous to someone trying to estimate the evolution
of the luminosity function from a photometric redshift sample (such
studies rarely include these correlated errors in their estimate of
their uncertainties). However, one can make an alternative set of
assumptions by modeling the distribution of the galaxies in the
three-dimensional spectral space as a pair of gaussians. For such a
model, and a model for the position and luminosity evolution of each
gaussian, one can calculate the probability of observing any given
galaxy. Thus, one could use the galaxy photometry to constrain the
luminosity evolution of galaxies in each gaussian, as well as the
evolution of the position of the gaussians. This method has its own
systematic errors and its own restrictive set of assumptions, but
provides an alternative path to calculating the luminosity function
directly from photometric redshifts.

\acknowledgments

MB was supported at the beginning of this work by the DOE and NASA
grant NAG 5-7092 at Fermilab. He is also grateful for the hospitality
of the Department of Physics and Astronomy at the State University of
New York at Stony Brook, who kindly provided computing facilities on
his frequent visits there. 
%MAS acknowledges the support of NSF grant AST-0071091.

The Sloan Digital Sky Survey (SDSS) is a joint project of The
University of Chicago, Fermilab, the Institute for Advanced Study, the
Japan Participation Group, the Johns Hopkins University, the
Max-Planck-Institute for Astronomy, New Mexico State University,
Princeton University, the United States Naval Observatory, and the
University of Washington. Apache Point Observatory, site of the SDSS
telescopes, is operated by the Astrophysical Research Consortium
(ARC).  Funding for the project has been provided by the Alfred
P.~Sloan Foundation, the SDSS member institutions, the National
Aeronautics and Space Administration, the National Science Foundation,
the U.~S.~Department of Energy, Monbusho, and the Max Planck
Society. The SDSS Web site is http://www.sdss.org/.
 
\begin{thebibliography}{DUM}
\bibitem[Binney \& Merrifield (1998)]{binney98a}
Binney, J., \& Merrifield, M.~1998, Galactic Astronomy (Princeton:
Princeton University Press)
\bibitem[Bruzual \& Charlot (1993)]{bruzual93a}
Bruzual, A.~G.,~\& Charlot, S.~1993, \apj, {405}, 538
\bibitem[Bud\'avari {\it et al.} (2000)]{budavari00a}
Budav\'ari, T.; Szalay, A. S.; Connolly, A. J.; Csabai, I.; Dickinson,
M.~(2000), \aj, 120, 1588
\bibitem[Csabai {\it et al.}~(2000)]{csabai00a}
Csabai, I., Connolly, A.~J., Szalay, A.~S., \& Budav\'ari,
T.~2000, \aj, 119, 69
\bibitem[Eisenstein {\it et al.}~(2001)]{eisenstein01a}
Eisenstein, D.~J., {\it et al.}~SDSS Collaboration~2001, 122, 2267
\bibitem[Fan (1999)]{fan99a}
Fan, X.~1999, \aj, 117, 2528
\bibitem[Frei \& Gunn (1994)]{frei94a}
Frei, Z., \& Gunn, J.~E.~1994, \aj, 108, 1476
\bibitem[Fukugita {\it et al.}~(1996)]{fukugita96a}
% SDSS Photometric
Fukugita, M., Ichikawa, T., Gunn, J.~E., Doi, M., Shimasaku, K., \&
Schneider, D.~P.~1996, \aj, 111, 1748
\bibitem[Fukugita, Shimasaku, \& Ichikawa (1995)]{fukugita95a}
% Galaxy colors in various photometric Band Systems
Fukugita, M., Shimasaku, K., \& Ichikawa, T.~1995, \pasp, 107, 945
\bibitem[Gunn {\it et al.}~(1998)]{gunn98a}
Gunn, J.~E., Carr, M.~A., Rockosi, C.~M., Sekiguchi, M., {\it et al.}~1998, \aj, 116, 3040
\bibitem[Hogg (1999)]{hogg99a}
Hogg, D.~W.~1999, astro-ph/9905116 
\bibitem[Oke \& Gunn (1983)]{oke83a}
Oke, J.~B., \& Gunn, J.~E.~1983, \apj, 266, 713
\bibitem[Oke \& Sandage (1968)]{oke68a}
Oke, J.~B., \& Sandage, A.~1968, \apj, 154, 21
\bibitem[Petrosian (1976)]{petrosian76a}
Petrosian, V.~1976, \apj, 209, L1
\bibitem[Press {\it et al.}~(1992)]{press92a}
Press, W.~H., Teukolsky, S.~A., Vetterling, W.~T., \& Flannery,
B.~P.~1992, Numerical Recipes (Cambridge: Cambridge Univ.~Press)
\bibitem[Schlegel, Finkbeiner \& Davis (1998)]{schlegel98a}
Schlegel, D.~J., Finkbeiner, D.~P., \& Davis, M.~1998, \apj, 500, 525
\bibitem[Schlegel {\it et al.} (2002)]{schlegel02a}
Schlegel, D.~J., {\it et al.}~2002, in preparation
\bibitem[Smith {\it et al.}~(2002)]{smith02a}
Smith, J.~A., {\it et al.}~SDSS Collaboration~2002, \aj, 123, 2121
\bibitem[Stoughton {\it et al.} (2002)]{stoughton02a}
Stoughton, C., {\it et al.}~2002, \aj, 123, 485
\bibitem[Strauss {\it et al.} (2002)]{strauss02a}
Strauss, M.~A., {\it et al.}~2002, submitted to \aj
\bibitem[SubbaRao {\it et al.} (2002)]{subbarao02a}
SubbaRao, M., {\it et al.}~2002, in preparation
\bibitem[York {\it et al.}~(2000)]{york00a}
York, D., {\it et al.}~2000, \aj, 120, 1579

\end{thebibliography}

\newpage

\include{tables}

\clearpage

\setcounter{thefigs}{0}

\clearpage
\stepcounter{thefigs}
\begin{figure}
\figurenum{\fignum}
\plotone{response_sdss.ps}
\caption{\label{response_sdss} Estimated filter response for all five
bands in the SDSS, as a function of observed wavelength, as measured
by Mamoru Doi. A 4 gigayear old instantaneous burst using the models
of \citet{bruzual93a} (and observed at $z=0$) is shown for reference.}
%at different redshifts. A 4 gigayear old instantaneous burst using the
%models of \citet{bruzual93a} is shown for reference. {\it Top panel:}
%The system consisting of \band{0.0}{u}, \band{0.0}{g}, \band{0.0}{r},
%\band{0.0}{i}, and \band{0.0}{z}.  {\it Bottom panel:} The system
%consisting of \band{0.0}{u}, \band{0.1}{g}, \band{0.1}{r},
%\band{0.1}{i}, and \band{0.2}{z}. We use this second system rather
%than the first because it requires less interpolation to determine
%\band{0.1}{g}, \band{0.1}{r}, and \band{0.1}{i}, and no extrapolation to
%determine \band{0.0}{u} and \band{0.1}{z}. 
\end{figure}

\clearpage
\stepcounter{thefigs}
\begin{figure}
\figurenum{\fignum}
\plotone{spur.ps}
\caption{\label{spur} The spectrum corresponding to the direction of
the spur in the upper right panel of Figure
\ref{k_coeffdist_plot}. Note the strong feature near 4000
\AA. Overplotted are the bandpasses for $\band{0.3}{g}$ and
$\band{0.3}{r}$. The strong features fall in the gap between the
bandpasses. Thus, in a linear fit to a galaxy at $z=0.3$, this
component can be used to fit the observed magnitudes without being
constrained to have reasonable behavior around 4000 \AA. }
\end{figure}

%\clearpage
%\stepcounter{thefigs}
%\begin{figure}
%\figurenum{\fignum}
%%\plotone{k_espec_plot.ps}
%\caption{\label{k_espec_plot} The four derived eigenspectra. Note that
%eigenspectra \#1, \#2, and \#3 are constrained to have zero total flux in the
%range between 3500\AA and 7500\AA. Eigenspectrum \#0 is not in any
%sense the ``average'' spectrum. }
%\end{figure}

\clearpage
\stepcounter{thefigs}
\begin{figure}
\figurenum{\fignum}
\plotone{k_coeffdist_plot.ps}
\caption{\label{k_coeffdist_plot} {\it Top panels}: Distribution of
the components of the four-parameter fit to the five-band SDSS
photometry for a random subsample consisting of 10,000 of the SDSS
galaxies. $a_0$ is linearly proportional to the flux between $3500\AA$
and $7000\AA$, while $a_1$, $a_2$, and $a_3$ contribute no flux in
this range. Thus, the ratios $a_1/a_0$, $a_2/a_0$, and $a_3/a_0$
describe the spectral type of the galaxy. $a_1/a_0$ is the most
variable parameter and thus is the better separator of galaxy
type. The spur extending from the upper left to the lower right from
the red dot in the $(a_3/a_0)$-$(a_1/a_0)$ plane is due to a
degeneracy for galaxies at $z\sim 0.3$, described in detail in Section
\ref{grgap}. {\it Bottom panel}: At fixed $a_2/a_0$ and $a_3/a_0$, the
inferred spectra corresponding to various values of $a_3/a_0$. Near
$a_3/a_0=-0.20$, the spectrum is similar to that of an elliptical
galaxy. For higher values, the spectrum becomes bluer. }
\end{figure}

\clearpage
\stepcounter{thefigs}
\begin{figure}
\figurenum{\fignum}
\plotone{k_model_plot.ps}
\caption{\label{k_model_plot} Reconstructed galaxy fluxes relative to
the observed galaxy fluxes, for all five SDSS bands, shown for a
random subsample consisting of around 10,000 of the SDSS galaxies. The
residuals are shown against redshift.  There is no systematic trend
with redshift in any band. The 5-$\sigma$ clipped estimate of the
scatter around the observed fluxes is listed for each band. In $u$,
$g$, $r$, and $i$ the scatter is consistent with the expected
photometric errors in the survey at all redshifts. At high redshift
the scatter in $z$ becomes large, most likely due to increasing
photometric errors. }
\end{figure}

\clearpage
\stepcounter{thefigs}
\begin{figure}
\figurenum{\fignum}
\plotone{main_colors_plot.z.ps}
\caption{\label{main_colors_plot.z} Color distributions in the
observed frame for SDSS Main Sample galaxies in the luminosity range
$-21.5<M_{\band{0.1}{r}}<-21.2$. This sample is complete (that is,
volume limited) for $0.05<z<0.17$. Left panels show the colors as a
function of redshift. Right panel shows the distributions of each
color at high and low redshift within the volume-limited
subsample. The observed colors clearly depend strongly on redshift,
even for the $\band{0.1}{u}$ band, where the low-redshift end is an
extrapolation of the data.}
\end{figure}

\clearpage
\stepcounter{thefigs}
\begin{figure}
\figurenum{\fignum}
\plotone{main_colors_plot.ps}
\caption{\label{main_colors_plot} Similar to Figure
\ref{main_colors_plot.z}, but now the colors are $K$-corrected to
$z=0.1$.  There is very little dependence of the colors on redshift,
even for the $\band{0.1}{u}$ band, where the low-redshift end is an
extrapolation of the data.}
\end{figure}

\clearpage
\stepcounter{thefigs}
\begin{figure}
\figurenum{\fignum}
\plotone{lrg_colors_plot.z.ps}
\caption{\label{lrg_colors_plot.z} Similar to Figure
\ref{main_colors_plot.z}, now showing LRG galaxies (Cut I) in the
luminosity range $-22.8<M_{\band{0.3}{r}}<-22.5$. Again, there is a
strong dependence on redshift.}
\end{figure}

\clearpage
\stepcounter{thefigs}
\begin{figure}
\figurenum{\fignum}
\plotone{lrg_colors_plot.ps}
\caption{\label{lrg_colors_plot} Same as Figure
\ref{lrg_colors_plot.z}, now $K$-correcting the LRG galaxies to
$z=0.3$.  The redshift dependence is greatly reduced for the LRGs in
comparison to Figure \ref{lrg_colors_plot.z}; on the other hand, there
are distinct trends of restframe color with redshift.  In
$\band{0.3}{(g-r)}$ an overall trend is apparent; LRGs at $z=0.4$ are
about 0.1 magnitudes bluer than LRGs at $z=0.2$. A change of this
magnitude is attributable to passive galaxy evolution, though
considerably more work needs to be done to show that this is
occurring. In $\band{0.3}{(r-i)}$, a blueward shift also occurs,
though at a much smaller level.}
\end{figure}

\clearpage
\stepcounter{thefigs}
\begin{figure}
\figurenum{\fignum}
\plotone{k_kcorrect_plot.ps}
\caption{\label{k_kcorrect_plot} $K$-corrections to $z=0.3$ as a
function of redshift in all five bands for a random subsample
consisting around 10,000 of the SDSS galaxies.  The range of
$K$-corrections at each redshift reflects the range of galaxy types at
each redshift.  The $K$-corrections are largest, and therefore the
most uncertain, for the \band{0.3}{u} and \band{0.3}{g} bands. While
we show the $K$-corrections for \band{0.3}{u} at $z<0.3$ and for
\band{0.3}{z} at $z>0.3$, and indeed these $K$-corrections are fairly
well-behaved, we do not recommend using these extrapolated results for
scientific purposes. Note the degeneracy at $z=0.3$.}
\end{figure}

\clearpage
\stepcounter{thefigs}
\begin{figure}
\figurenum{\fignum}
\plotone{k_speck_plot.fitfib.0.1.ps}
\caption{\label{k_speck_plot.fitfib.0.1} Difference between the
$K$-corrections to $z=0.1$ determined from the spectroscopy and those
determined from the analysis of broad-band magnitudes {\it
synthesized} from the same spectra.}
\end{figure}

\clearpage
\stepcounter{thefigs}
\begin{figure}
\figurenum{\fignum}
\plotone{compareci.ps}
\caption{\label{compareci} Difference in the $K$-corrections to
$z=0.1$ in each band between the method used in Figure
\ref{k_kcorrect_plot} and the method of simply interpolating between
adjacent bands fitting a power-law SED. The differences are small in
\band{0.1}{r}, \band{0.1}{i}, and \band{0.1}{z}, where galaxy SEDs
have simple shapes. There are large systematic differences in
\band{0.1}{u} and \band{0.1}{g}, for which the 4000 \AA\ break is
important in the spectral templates used. }
\end{figure}

\clearpage
\stepcounter{thefigs}
\begin{figure}
\figurenum{\fignum}
\plotone{comparecibreak.ps}
\caption{\label{comparecibreak} Same as Figure \ref{comparecibreak},
now comparing to a method of interpolating the bandpasses using
power-laws, and fitting for the 4000 \AA break. The systematic trends
in \band{0.1}{u} and \band{0.1}{g} are gone (though there is
considerable scatter in \band{0.1}{u}). }
\end{figure}

\clearpage
\stepcounter{thefigs}
\begin{figure}
\figurenum{\fignum}
\plotone{k_speck_plot.0.1.ps}
\caption{\label{k_speck_plot.0.1} Difference between the
$K$-corrections to $z=0.1$ determined from the spectroscopy and those
determined from the analysis of the broad-band Petrosian magnitudes, for
Main Sample galaxies.}
\end{figure}

\clearpage
\stepcounter{thefigs}
\begin{figure}
\figurenum{\fignum}
\plotone{k_speck_plot.0.3.ps}
\caption{\label{k_speck_plot.0.3} Difference between the
$K$-corrections to $z=0.3$ determined from the spectroscopy and those
determined from the analysis of the broad-band model magnitudes, for
LRGs.}
\end{figure}


\end{document}
