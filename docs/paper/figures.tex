\clearpage

\setcounter{thefigs}{0}

\clearpage
\stepcounter{thefigs}
\begin{figure}
\figurenum{\fignum}
\plotone{response_sdss.ps}
\caption{\label{response_sdss} Estimated filter response for all five
bands in the SDSS, as a function of observed wavelength. A 4 gigayear
old instantaneous burst using the models of \citet{bruzual93a} (and
observed at $z=0$) is shown for reference.}
%at different redshifts. A 4 gigayear old instantaneous burst using the
%models of \citet{bruzual93a} is shown for reference. {\it Top panel:}
%The system consisting of \band{0.0}{u}, \band{0.0}{g}, \band{0.0}{r},
%\band{0.0}{i}, and \band{0.0}{z}.  {\it Bottom panel:} The system
%consisting of \band{0.0}{u}, \band{0.1}{g}, \band{0.1}{r},
%\band{0.1}{i}, and \band{0.2}{z}. We use this second system rather
%than the first because it requires less interpolation to determine
%\band{0.1}{g}, \band{0.1}{r}, and \band{0.1}{i}, and no extrapolation to
%determine \band{0.0}{u} and \band{0.1}{z}. 
\end{figure}

%\clearpage
%\stepcounter{thefigs}
%\begin{figure}
%\figurenum{\fignum}
%%\plotone{k_espec_plot.ps}
%\caption{\label{k_espec_plot} The four derived eigenspectra. Note that
%eigenspectra \#1, \#2, and \#3 are constrained to have zero total flux in the
%range between 3500\AA and 7500\AA. Eigenspectrum \#0 is not in any
%sense the ``average'' spectrum. }
%\end{figure}

\clearpage
\stepcounter{thefigs}
\begin{figure}
\figurenum{\fignum}
\plotone{k_coeffdist_plot.ps}
\caption{\label{k_coeffdist_plot} {\it Top panels}: Distribution of
the components of the four-parameter fit to the five-band SDSS
photometry for a random subsample consisting of 10,000 of the SDSS
galaxies. $a_0$ is linearly proportional to the flux between $3500\AA$
and $7000\AA$, while $a_1$, $a_2$, and $a_3$ contribute no flux in
this range. Thus, the ratios $a_1/a_0$, $a_2/a_0$, and $a_3/a_0$
describe the spectral type of the galaxy. $a_1/a_0$ is the most
variable parameter and thus is the better separator of galaxy
type. {\it Bottom panel}: At fixed $a_2/a_0$ and $a_3/a_0$, the
inferred spectra corresponding to various values of $a_3/a_0$. Near
$a_3/a_0=-0.20$, the spectrum is similar to that of an elliptical
galaxy. For higher values, the spectrum becomes bluer. }
\end{figure}

\clearpage
\stepcounter{thefigs}
\begin{figure}
\figurenum{\fignum}
\plotone{k_model_plot.ps}
\caption{\label{k_model_plot} Reconstructed galaxy fluxes relative to
the observed galaxy fluxes, for all five SDSS bands, shown for a
random subsample consisting of around 10,000 of the SDSS galaxies. The
residuals are shown against redshift.  There is no systematic trend
with redshift in any band. The 5-$\sigma$ clipped estimate of the
scatter around the observed fluxes is listed for each band. In $g$,
$r$, and $i$ the scatter is consistent with the expected photometric
errors in the survey. In $z$ and $u$ there is more scatter. In the
case of $z$ this is attributable to Poisson noise in the
observations. In the case of $u$, it is not, but it may simply be that
there are more intrinic differences between galaxies in the $u$-band
than elsewhere.}
\end{figure}

\clearpage
\stepcounter{thefigs}
\begin{figure}
\figurenum{\fignum}
\plotone{k_coeff_plot2.ps}
\caption{\label{k_coeff_plot1} }
\end{figure}

\clearpage
\stepcounter{thefigs}
\begin{figure}
\figurenum{\fignum}
\plotone{k_coeff_plot1.ps}
\caption{\label{k_coeff_plot1} }
\end{figure}

\clearpage
\stepcounter{thefigs}
\begin{figure}
\figurenum{\fignum}
%\plotone{kcorrect.sample8b15.ps}
\caption{\label{kcorrect.sample8b15} $K$-corrections as a function of
redshift in all five bands for a random subsample consisting of 8,000
of the SDSS galaxies. The $K$-corrections are largest, and therefore
the most uncertain, for the \band{0.0}{u} and \band{0.1}{g} bands. The
range of $K$-corrections at each redshift reflects the range of galaxy
types at each redshift.}
\end{figure}

\clearpage
\stepcounter{thefigs}
\begin{figure}
\figurenum{\fignum}
%\plotone{ciCompare.sample8b15.ps}
\caption{\label{ciCompare.sample8b15} Difference in the
$K$-corrections in each band between the method used in Figure
\ref{kcorrect.sample8b15} and the method of simply interpolating
between adjacent bands fitting a power-law SED. The differences are
small in \band{0.1}{r}, \band{0.1}{i}, and \band{0.2}{z}, where galaxy
SEDs have simple shapes. There are large differences in \band{0.0}{u}
and \band{0.1}{g}, for which the 4000 \AA\ break is important. Note
particularly the systematic differences in \band{0.1}{g} with
redshift.}
\end{figure}

\clearpage
\stepcounter{thefigs}
\begin{figure}
\figurenum{\fignum}
%\plotone{cibreakCompare.sample8b15.ps}
\caption{\label{cibreakCompare.sample8b15} Same as Figure
\ref{ciCompare.sample8b15}, now comparing the $K$-corrections of Figure
\ref{kcorrect.sample8b15} with the ``interpolation with a break''
method. This method fits a power law between adjacent bands, except at
4000 \AA, where we fit for the size of the 4000\AA\ break (assuming
that $f(\lambda)\propto \lambda^2$ for $\lambda<4000$ \AA). This
greatly improves the agreement in \band{0.1}{g} while making the
disagreement in \band{0.0}{u} only slightly worse. These results
indicate that it important to account for the structure in the blue
region of the spectrum when performing $K$-corrections.}
\end{figure}

\clearpage
\stepcounter{thefigs}
\begin{figure}
\figurenum{\fignum}
%\plotone{specK.ps}
\caption{\label{specK} Similar to Figure \ref{ciCompare.sample8b15},
now comparing the $K$-corrections of Figure \ref{kcorrect.sample8b15}
with $K$-corrections determined from the spectra (which can only be
calculated for the \band{0.1}{g}, \band{0.1}{r}, and \band{0.1}{i}
bands). In all bands, the $K$-corrections are very similar, giving us
confidence in our results. This result is actually remarkable more for
what it says about the high quality of the spectrophotometry in the
SDSS survey.}
\end{figure}

\clearpage
\stepcounter{thefigs}
\begin{figure}
\figurenum{\fignum}
%\plotone{aperturevsz.M.sample8b15.ps}
\caption{\label{aperturevsz.M.sample8b15} }
\end{figure}

